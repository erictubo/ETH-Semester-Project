\chapter{Introduction}
\label[chapter]{chapter:introduction}

% Motivation
Obstacle detection is crucial for safe operation of railway vehicles. A prerequisite for this is identifying a region of interest (ROI), in other words knowing where to look for obstacles, which means that the tracks ahead of the vehicle need to be correctly identified and located. This could be done by projecting a known railway map into camera view. However, this requires precise knowledge of the camera position and rotation -- not typically available in the field or with existing datasets. Given the degree of accuracy required for long-range obstacle detection (LROD), this is a non-trivial task.

% Problem Description
The aim of this semester project is to develop an extrinsic camera calibration and reprojection pipeline based on visual cues and map information, where the intrinsics of the camera are known. The available data includes a set of images with associated pose readings from a GPS sensor, as well as a global railway map. The images are taken from a camera that is mounted to the front of a vehicle, which is driving along the railway track. The GPS sensor is also attached to the vehicle, but in a different place. Moreover, the map information includes OpenStreetMap (OSM) data with the positions and properties of railway nodes and tracks, as well as elevation data.

% TODO: add figure with data types

% TODO: add figure with setup

% Related Work
A previous solution \cite{spiegelhalter2023extrinsics} proposes a geometric approach, applying a line-detection algorithm to extract image features such as vanishing points, railway sleepers, and poles from an image, which are used to compute the camera pose step-by-step. However, this approach is not generalizable as it only works on carefully-selected individual frames with straight tracks.

% My Solution: what makes it different
The solution presented in this report proposes to directly reproject the 3D map into camera view, and formulate an optimization problem to find the correct camera pose. The aim is to build a robust pipeline that works across multiple frames, thus alleviating the prior shortcomings.

This report is structured as follows. \Cref{chapter:background} provides background information in order to introduce relevant technical concepts for an improved understanding. \Cref{chapter:method} outlines the proposed method, describing all main components, processes, as well as inputs and outputs in detail. \Cref{chapter:implementation} discusses the implementation details, which includes the software architecture, methods, and data types so that others are able to operate and extend the pipeline. \Cref{chapter:results} presents and evaluates the results. \Cref{chapter:conclusion} concludes the report and discusses possible future work.