\chapter{Results \& Evaluation}
\label{chapter:results}

This chapter presents the results of the proposed method, and evaluates its performance in terms of accuracy, robustness, and generalization. Focus is put onto the single-frame optimization in \Cref{sec:single-frame}, but the adaptation to multi-frame optimization is also discussed in \Cref{sec:multi-frame}. In both cases, the results are evaluated regarding different aspects and recommendations are given to optimize performance.


\section{Single-frame}
\label[section]{sec:single-frame}

This section evaluates single-frame optimization with respect to convergence, robustness, and accuracy on a variety of different scenes, using the same initial camera pose. The converged results are shown in \Cref{fig:single-frame_final_correspondences}.

\begin{figure}[h]
    \centering
    \begin{minipage}[t]{0.48\textwidth}
        \includegraphics[width=\textwidth]{images/0490_optimization_final.png}
    \end{minipage}
    \hfill
    \begin{minipage}[t]{0.48\textwidth}
        \includegraphics[width=\textwidth]{images/0570_optimization_final.png}
    \end{minipage}
    \vskip\baselineskip
    \begin{minipage}[t]{0.48\textwidth}
        \includegraphics[width=\textwidth]{images/0950_optimization_final.png}
    \end{minipage}
    \hfill
    \begin{minipage}[t]{0.48\textwidth}
        \includegraphics[width=\textwidth]{images/5410_optimization_final.png}
    \end{minipage}
    \caption{Converged correspondences after single-frame optimization.}
    \label[figure]{fig:single-frame_final_correspondences}
\end{figure}

It can be seen that the final alignment between reprojected and observed tracks is quite good in all cases. Even if the tracks are not perfectly aligned, as is the case with some intersections (e.g. bottom right image) which is likely due to inaccuracies in the railway map, the track directions on all sides of the intersection is still very well aligned.

\subsection*{Convergence \& Initial Estimates}

The method converges predictably and reliably, given a decent initial height and depth estimate. During analysis, the method always converged after 10-50 ICP iterations (i.e. number of correspondence updates), depending on the scene complexity and initial pose estimate. In this context, convergence implies that the alignment does no longer noticeable improve with further iterations.

However, it is important to have a close initial height and depth position estimate, which are both hard to optimize for. Height is difficult to optimize for because a slight change ($\pm$ 1m) leads to a very different reprojection that may not converge to the global minimum. Depth, on the other hand, is difficult to optimize for because the reprojection error is very flat in this dimension, i.e. a small change in depth hardly leads to any change in the reprojection, since the track does not change as quickly with depth. Nevertheless, the height and depth positions are easy to approximate beforehand.

When it comes to all other initial guesses, including the lateral position and all rotations, they were simply set to zero in the camera frame -- implying that the camera faces perfectly forward at the same lateral position as the GPS sensor -- without any convergence issues. Overall, the method is resilient to incorrect initial guesses, as long as the height and depth estimates are decent.

\subsection*{Robustness \& Generalization}

The method is also robust to scene changes and can be generalized to new data without problems. Even if the initial ICP correspondences are predominantly incorrect, i.e. the wrong tracks are associated with each other, they still adjust themselves during optimization. An example of this is shown in \Cref{fig:self-adjusting_correspondences}. This self-correcting of correspondences to the correct tracks ocurrs relatively quickly within the first few iterations, after which the pose is only refined.

\begin{figure}[h]
    \centering
    \begin{minipage}[t]{0.48\textwidth}
        \includegraphics[width=\textwidth]{images/0950_optimization_initial.png}
    \end{minipage}
    \hfill
    \begin{minipage}[t]{0.48\textwidth}
        \includegraphics[width=\textwidth]{images/0950_optimization_final.png}
    \end{minipage}
    \caption{Initially incorrect correspondences (left) adjusting themselves (right).}
    \label[figure]{fig:self-adjusting_correspondences}
\end{figure}

Generalization to different scenes and new datasets is also possible, yet keyframes with higher track complexity should be selected to achieve better results. Higher track complexity means that the scenes should have a depth-variability of the track, as is the case in curves or intersections, as opposed to simple straight tracks. Examples of such scenes are the chosen frames in this chapter. In general, more complex tracks also lead to better results because more information is available to optimize for in different dimensions. With these prerequisites, the method is robust and generalizes well.

\subsection*{Accuracy \& Precision}

The accuracy of the pose estimation can be evaluated using the stereo camera calibration data, which provides a ground truth for the relative pose between the stereo cameras. At first, the stereo cameras are optimized separately, using the same optimization method as for the single-frame optimization, to get their poses relative to the GPS sensor. Then, the relative pose between the stereo cameras is computed and compared to the calibration ground truth.

\begin{table}[h]
    \caption{Stereo calibration compared to relative camera transformation obtained from different frames.}
    \centering
    \begin{tabular}{ | l | c | c | c | }
        \hline
            & $\Delta X$ [m]& $\Delta Y$ [m] & $\Delta Z$ [m] \\
        \hline
        \textbf{Calibration}    & 0.307 & 0.002  & 0.010 \\
        \hline
        \textbf{Frame 1}        & 0.289 & -0.006 & 0.163 \\
        \textbf{Frame 2}        & 0.261 & -0.047 & 0.128 \\
        \textbf{Frame 3}        & 0.300 & 0.010  & 0.133 \\
        \hline
    \end{tabular}
    \label[table]{tab:stereo_calibration_accuracy}
\end{table}

As can be seen from \Cref{tab:stereo_calibration_accuracy}, the accuracy of the calculated stereo camera transformations is relatively high. The horizontal offset $\Delta X$ and vertical offset $\Delta Y$ are both within less than 0.05 m of the calibration value, while there's a slightly higher error of up to 0.16 m in the depth offset $\Delta Z$. The horizontal and vertical offset accuracies reflect the accuracy of the reprojection and optimization pipeline. Moreover, the results are very consistent, with all three frames producing similar results, which indicates that the pose estimation is very precise as well.

The lack of accuracy in the depth estimate can be explained by the fact that the track does not change very quickly with depth (at least not on a centimeter-scale), which makes it hard to accurately estimate and sensitive to the initial guess. Nevertheless, for the same reason this is also the least critical dimension to estimate accurately, since small errors don't lead to visual reprojection errors.

When it comes to rotation, the results are all quite accurate and close to zero, as expected. If this were not the case, the reprojections would look very different since small angular errors can lead to large reprojection errors.

\vfill

\section{Multi-frame}
\label[section]{sec:multi-frame}

The multi-frame optimization is evaluated on two sets of 3 frames, one with consecutive frames and another with keyframes of different scenes. All frames use the same initial camera pose estimate, as outlined in single-frame optimization above.

\subsection*{Multiple Keyframes}

The first set of frames consists of three keyframes with different scenes that are optimized simultaneously. The initial estimates are shown in \Cref{fig:multi-frame_initial_correspondences}, while the converged outputs are shown in \Cref{fig:multi-frame_final_correspondences}. A comparison to the equivalent single-frame optimization outputs is shown in \Cref{fig:equivalent_single-frame_final_correspondences}.

\begin{figure}[h]
    \centering
    \begin{minipage}[t]{0.32\textwidth}
        \includegraphics[width=\textwidth]{images/0490_optimization_initial_crop.jpg}
    \end{minipage}
    \hfill
    \begin{minipage}[t]{0.32\textwidth}
        \includegraphics[width=\textwidth]{images/0570_optimization_initial_crop.jpg}
    \end{minipage}
    \hfill
    \begin{minipage}[t]{0.32\textwidth}
        \includegraphics[width=\textwidth]{images/0950_optimization_initial_crop.jpg}
    \end{minipage}
    \caption{Initial correspondences between reprojected and observed points.}
    \label[figure]{fig:multi-frame_initial_correspondences}
\end{figure}

\begin{figure}[h]
    \centering
    \begin{minipage}[t]{0.32\textwidth}
        \includegraphics[width=\textwidth]{images/0490_multi-frame_final_crop.jpg}
    \end{minipage}
    \hfill
    \begin{minipage}[t]{0.32\textwidth}
        \includegraphics[width=\textwidth]{images/0570_multi-frame_final_crop.jpg}
    \end{minipage}
    \hfill
    \begin{minipage}[t]{0.32\textwidth}
        \includegraphics[width=\textwidth]{images/0950_multi-frame_final_crop.jpg}
    \end{minipage}
    \caption{Converged correspondences after multi-frame optimization.}
    \label[figure]{fig:multi-frame_final_correspondences}
\end{figure}

\begin{figure}[h]
    \centering
    \begin{minipage}[t]{0.32\textwidth}
        \includegraphics[width=\textwidth]{images/0490_optimization_final_crop.jpg}
    \end{minipage}
    \hfill
    \begin{minipage}[t]{0.32\textwidth}
        \includegraphics[width=\textwidth]{images/0570_optimization_final_crop.jpg}
    \end{minipage}
    \hfill
    \begin{minipage}[t]{0.32\textwidth}
        \includegraphics[width=\textwidth]{images/0950_optimization_final_crop.jpg}
    \end{minipage}
    \caption{Equivalent converged correspondences after single-frame optimization.}
    \label[figure]{fig:equivalent_single-frame_final_correspondences}
\end{figure}

When looking at \Cref{fig:multi-frame_final_correspondences}, the multi-frame results don't have a very good alignment. It seems that there are conflicts in the data, since the same camera pose estimate leads to inconsistent reprojections across frames. This becomes especially aparent when comparing to the equivalent single-frame optimization results in \Cref{fig:equivalent_single-frame_final_correspondences}, which are much more consistent.

\subsection*{Consecutive Frames}

To test if these issues persist with more similar frames, a second set of frames is used that consists of three nearby frames with a similar scene that are optimized simultaneously. The initial estimates are shown in \Cref{fig:nearby_multi-frame_initial_correspondences}, while the converged outputs are shown in \Cref{fig:nearby_multi-frame_final_correspondences}.

\begin{figure}[h]
    \centering
    \begin{minipage}[t]{0.32\textwidth}
        \includegraphics[width=\textwidth]{images/1900_nearby_multi-frame_initial_crop.jpg}
    \end{minipage}
    \hfill
    \begin{minipage}[t]{0.32\textwidth}
        \includegraphics[width=\textwidth]{images/1905_nearby_multi-frame_initial_crop.jpg}
    \end{minipage}
    \hfill
    \begin{minipage}[t]{0.32\textwidth}
        \includegraphics[width=\textwidth]{images/1910_nearby_multi-frame_initial_crop.jpg}
    \end{minipage}
    \caption{Initial correspondences between reprojected and observed points.}
    \label[figure]{fig:nearby_multi-frame_initial_correspondences}
\end{figure}

\begin{figure}[h]
    \centering
    \begin{minipage}[t]{0.32\textwidth}
        \includegraphics[width=\textwidth]{images/1900_nearby_multi-frame_final_crop.jpg}
    \end{minipage}
    \hfill
    \begin{minipage}[t]{0.32\textwidth}
        \includegraphics[width=\textwidth]{images/1905_nearby_multi-frame_final_crop.jpg}
    \end{minipage}
    \hfill
    \begin{minipage}[t]{0.32\textwidth}
        \includegraphics[width=\textwidth]{images/1910_nearby_multi-frame_final_crop.jpg}
    \end{minipage}
    \caption{Converged correspondences after nearby multi-frame optimization.}
    \label[figure]{fig:nearby_multi-frame_final_correspondences}
\end{figure}

While the alignment in \Cref{fig:nearby_multi-frame_final_correspondences} seems slightly better than in \Cref{fig:multi-frame_final_correspondences}, it is also not very precise. The same issues with inconsistent reprojections across frames are still present. The fact that even nearby frames lead to inconsistent reprojections is a strong indicator of the data being too imprecise.

\newpage
\subsection*{Analysis}

From the results presented in this section, it has become clear that the data is not very precise across frames to be optimized together for consistent reprojections. The measurements tend to be somewhat noisy, which leads to inconsistent reprojections across frames. This is likely due to a combination of different factors, including RTK-GPS sensor noise, elevation data, and the railway nodes.

While the RTK-GPS sensor is very accurate globally, it is not always precise locally. \Cref{fig:GPS_position_measurements} shows RTK-GPS position measurements, while \Cref{fig:GPS_rotation_measurements} shows RTK-GPS rotation measurements.

\begin{figure}[h]
    \centering
    \includegraphics[width=0.85\textwidth]{images/GPS_position_measurements.png}
    \caption{GPS position measurements (black) compared to railway nodes (red).}
    \label[figure]{fig:GPS_position_measurements}
\end{figure}

\begin{figure}[h]
    \centering
    \includegraphics[width=0.95\textwidth]{images/GPS_rotation_measurements.png}
    \caption{GPS rotation measurements for roll (red), pitch (green), and yaw (blue).}
    \label[figure]{fig:GPS_rotation_measurements}
\end{figure}

The position measurements seem very precise, since they all lie along a continuous line that follows the railway track. However, it could be that there are longitudinal jumps between frames, meaning that the data might not be as precise in measuring the position along the track. This could explain some inconsistencies in \Cref{fig:nearby_multi-frame_final_correspondences}, where the position does not seem to be consistent with the reprojected tracks.

When it comes to RTK-GPS rotation measurements, the data is also not very precise. In all three dimensions it varies by up to 0.5 degrees between frames while it should effectively stay the same on this straight track segment. Combining this small error in all three dimensions and considering the fact that small rotations can lead to large reprojection errors, it is likely a big factor in the inconsistent reprojections. This would explain the inconsistencies in \Cref{fig:multi-frame_final_correspondences}, where the rotation does not match the reprojections.

Comparing the RTK-GPS height measurements (blue) with the local elevation data (red) in \Cref{fig:GPS_height_measurements}, it appears that the former are very precise, while the latter varies by up to 0.2 m across frames. Yet, small errors in the elevation data (used for creating the 3D point clouds per railway track) are unlikely to be a big factor in the inconsistent reprojections, since all frames are affected equally. Also note that, for the RTK-GPS height measurements, there have been some outliers in the data (not visible in the graph), which are likely due to bad signal or else. Avoiding these outliers is important, but otherwise the GPS height measurements do not seem to be the issue.

\begin{figure}[h]
    \centering
    \includegraphics[width=0.7\textwidth]{images/GPS_height_measurements.png}
    \caption{GPS height measurements (blue) vs. local elevation data (red).}
    \label[figure]{fig:GPS_height_measurements}
\end{figure}

Finally, the railway nodes from the OpenStreetMap data are also a possible source of error. However, similar to the elevation data they are unlikely to be a big factor in the inconsistent reprojections, since all frames are affected equally. \Cref{fig:GPS_position_measurements} also shows that the railway nodes look relatively precise, in addition to being globally accurate.

In conclusion, if the above data sources were more precise, the reprojections would be more consistent across frames. This would lead to better results when optimizing multiple frames together.

\subsection*{Recommendations}

In order to overcome the limitations of the data and obtain more precise state estimates, it would be very useful to implement a sensor fusion algorithm, such as a Kalman Filter. This means combining the RTK-GPS sensor with available IMU and odometry data. The RTK-GPS sensor is very accurate globally, but not very precise locally. The IMU and odometry are precise locally, but drift over time. By combining all three, it is possible to obtain a more precise state estimate that is both precise locally and globally. This would lead to reduced reprojection errors across frames.

A low number of only 3 frames has purely been chosen for evaluation purposes, since this made it easier to visualize the results. In practice, the method can be applied to a much larger number of frames, which would merely take longer to optimize but also not necessarily lead to better results. Using a selection of about 5-20 different keyframes is likely a good balance between accuracy and optimization time.