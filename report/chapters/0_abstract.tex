\chapter*{Abstract}
\addcontentsline{toc}{chapter}{Abstract}

% Motivation & Problem Description
Vision-based driver assistance systems play a critical role in improving the safety of railway vehicles, especially when it comes to collision detection. In this work, we propose an online method for extrinsic calibration of a camera mounted to a track vehicle across multiple frames. The method makes use of a preinstalled GPS sensor to localize the vehicle with respect to a map and a detail camera to identify obstacles between the railway tracks. In order to determine this region of interest, local railway tracks from the map are reprojected into camera view, requiring precise knowledge of the camera pose.

% Solution: method
Our proposed method to estimate the 6 degree-of-freedom camera pose is based on an optimization approach. The railway map and elevation data are processed in order to reproject local railway tracks into camera view for each frame, upon which the error between reprojected and detected tracks in the image is minimized, utilizing an iterative closest points (ICP) algorithm.

% Conclusion: evaluation
Evaluation across a variety of scenes shows that the proposed method is able to accurately and precisely estimate the camera pose, as compared to stereo camera calibration data. Given a decent initial height and depth estimate, it is robust to different track shapes -- keyframes with depth-variability of the track actually improve the result. When optimizing across multiple frames, the accuracy of the method is limited by sensor noise, in particular the RTK-GPS rotation measurements, where small angular errors lead to reprojection inconsistencies. To overcome this limitation, sensor fusion can be applied to improve the GPS state estimate with IMU and odometry data.