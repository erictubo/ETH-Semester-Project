\chapter{Implementation}
\label{chapter:implementation}

This chapter dives into the implementation details of the method described in the previous chapter



The pipeline is designed to be modular, such that each component can be replaced by a different implementation, as long as the inputs and outputs are compatible. This allows for easy experimentation with different approaches, as well as the possibility to use the pipeline in different contexts.


Objects, classes \& interactions

Algorithmic implementation, efficiency, speed

Flowchart of code (files, classes, methods)


Better as table ???



Using Python for most tasks

C++ for optimization with Ceres

Libraries: OpenCV, NumPy, Ceres, ...

\section{Python Classes \& Objects (Methods, Data)}


... main file \& sequence


\subsection{Railway}

Which methods \& data types enable the process as described in method


For efficiency: build using combined map of relevant keyframes


\subsection{Keyframe \& GPS}


Image, annotations


\subsection{Camera}





\subsection{Transformation}






\section{C++ Optimization: Ceres Solver}

\subsection{Cost Function}

\subsection{Residuals}

\subsection{Parameters}



\section{Required Input Data}

Data file + any files imported and exported

Add to ReadMe: where to specify file paths / how to get data


railway map data ... from OSM file ?

elevation data ... from file ?



images \& poses ... export from ROS Bags



Annotations as CSV ... using Website ? to create annotations

saving Railway object to file


\section{Output Data}

File paths to specify to save visualizations etc.