\chapter{Introduction}
\label{sec:introduction}
%\chapter{Einleitung}
%\label{sec:einleitung}

With the increasing amount of traffic in cities, the demand for safe and frequent public transport has grown significantly \citep{dell2011quality}. This fuels the interest in having a vision-based driver support system for tramways and trains, as they improve safety with collision assessment protocols and can, therefore, potentially even lead to full autonomy. This development could additionally deem itself more profitable. With the demanded higher frequency of trains, no additional drivers would be required, and the cameras used are low cost and can easily be integrated into existing \text{infrastructures} \citep{trentesaux2018autonomous}.\\

To ensure the successful implementation of self-driving track vehicles, collision prevention is a key priority. This requires looking as far ahead as possible along the tracks, as the long braking distances necessitate early detection of obstacles. The region of interest (ROI) is consequently the area in between the railway tracks ahead of the track vehicle. Finding the ROI can be achieved by either detecting the railway tracks in every frame from the bottom of the image frame to as far ahead as possible or by projecting the known map of railway tracks into the image. The first method must be able to handle, e.g., illumination changes and occlusions, whereas the second method doesn't have these limitations.\\ 

In this thesis, we focus on the second method, where knowing the camera's pose at the front of the track vehicle is crucial for an accurate projection of the railway tracks. However, determining the extrinsics is complicated because cameras are often detached and reattached differently due to the lack of mounting fixtures, and an on-site calibration process is deemed cumbersome. This means a fast online computation of the camera's relative position and relative orientation with respect to the global positioning system (GPS) after the camera has been newly attached to a track vehicle is needed. It is given that the GPS is already preinstalled on every vehicle for self-localization. \\

This paper assumes that a single camera is attached to the train, which has already been intrinsically calibrated. Assuming that the monocular camera and GPS are installed on one rigid body, the extrinsics, once computed, can continuously be used until the camera is dismounted from the train. There are three main approaches to finding the extrinsics of a monocular camera. The first approach relies on instruments such as calibration boards or markers placed in the surroundings of the camera \citep{wang2015inverse}. This was dismissed as we wanted to use the visual information collected during the drive.\\

The second approach trains a convolutional neural network to determine the extrinsics of the camera \citep{lin20203d}. This is not feasible in our case due to the lack of labeled data for track vehicles.\\

The third approach leverages the geometry of the image, a process called photogrammetry. Here some authors made use of the three vanishing points present in an image to calibrate the camera \citep{caprile1990using} \citep{brauer2001image} \citep{grammatikopoulos2007automatic} \citep{cipolla1999camera}. This only works well if all vanishing points lie in the finite space from the image. This isn't the case in our visual information, but inspiration was taken out of this approach.\\

We propose a method that estimates the six parameters of the pose in a multi-staged process, where some leverage the location of one of the horizontal vanishing points visible in the image \cite{namazi2022geolocation} and the others utilize the visual detection of distinctive landmarks. The detection of the used horizontal vanishing point is rather complicated in, e.g., construction sites but not so much in track scenes, as it lies at the intersection of the linearly extended railway tracks. The computed estimates are averaged over several frames to account for the potential slight lateral movement of the train due to uneven railway tracks. Validated on the alignment of the projected railway tracks to the tracks in the image, applying the proposed method led to a large improvement. This method has the advantage of also being applicable to existing data sets. \\

The thesis is structured by first giving background information on the theory used. It is followed by a detailed approach on how the parameters are estimated in \text{chapter 3}. The results of the proposed method are described in \text{chapter 4} and discussed in the last chapter, where also an outlook is given.
